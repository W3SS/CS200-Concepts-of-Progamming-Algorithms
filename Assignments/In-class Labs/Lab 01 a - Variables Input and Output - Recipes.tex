\documentclass[a4paper,12pt]{book}
\usepackage[utf8]{inputenc}
\title{}
\author{Rachel Morris}
\date{\today}

\usepackage{rachwidgets}
\usepackage{fancyhdr}
\usepackage{lastpage}
\usepackage{boxedminipage}

\pagestyle{fancy}
\fancyhf{}
\lhead{Recipes Lab}
\chead{ }
\rhead{ }
\rfoot{\thepage\ of \pageref{LastPage}}
\lfoot{By Rachel Morris \date}

\renewcommand{\headrulewidth}{2pt}
\renewcommand{\footrulewidth}{1pt}

\begin{document}

    \chapter*{Lab 1: Recipe} \stepcounter{chapter}
        \section*{Information}

                \paragraph*{ Topics } ~\\
                    \begin{itemize}
                        \item Building a basic program
                        \item Variables
                        \item Data Types
                        \item Output with cout
                        \item Input with cin
                    \end{itemize}
                    
                \paragraph*{ Turn in } ~\\
                    Once you are finished, turn in each \textbf{ part }'s source file. These will
                    have the extention .cpp, and in Windows, it will show up as \textbf{ C++ source file }.

        \section*{ Program 1: Output only }
            \subsection*{ Create the project }

% side box %
\begin{WrapTextSide}
    IDE means \\ Integrated Development Environment
\end{WrapTextSide}
% side box %
            
                First, you will need to create a new project in your IDE (Visual Studio, Code::Blocks, etc.).
                Name the project \textbf{ lab1\_yourname }. You will also create one source file,
                name it \textbf{ lab1\_part1.cpp } \\

                For help on how to create a project and a file, see below.

                
            
% introduction %
\begin{mdframed}[backgroundcolor=intro] 
\textbf{ Creating a project in Visual Studio } ~\\

\begin{enumerate}
    \item Go to \textbf{ File - New - Project... }.
    \item In the New Project window, select \textbf{ Visual C++ } from the left side.
    \item In the center, select \textbf{ Empty Project }. (It must be Empty Project!)
    \item At the bottom, give your project a name in the \textbf{ Name } textbox.
    \item Next to the \textbf{ Location } textbox, click on \textbf{ Browse... }
    \begin{itemize}
        \item If you're working on a school computer, it might be good to
        navigate to the Desktop and place your project here for quick access.
        \item Avoid putting your project in the \textit{ temp } directory.
        \item If you store your project on a thumbdrive, the compile process
        will be very slow. Move the project to your thumbdrive after you're done
        with your work.
    \end{itemize}
    \item Click OK
\end{enumerate}

At this point, you will have a new project, but it will be empty. You will need to add a source file.

\begin{enumerate}
    \item In the \textbf{ Solution Explorer }, right click your project and select
        \textbf{ Add - New Item... }.
    \item Select \textbf{ C++ File (.cpp) }, and at the bottom of the window,
        enter a name in the \textbf{ Name } textbox.
    \item Click \textbf{ Add }.
\end{enumerate}

Now when you double-click the source file in the Solution Explorer, it will open up.

\end{mdframed}
% introduction %

            \newpage
            \subsection*{ Add libraries }
            
% side box %
\begin{WrapTextSide}
    A \textbf{ library } is code that has been written by somebody else,
    and ``packaged" so that the code can be used across multiple programs.
\end{WrapTextSide}
% side box %

               To display text to the screen, we will need to include
               the \textbf{ iostream } library in our program. The
               iostream library contains a command called \texttt{ cout }
               which allows us to do this.

               At the top of your source file, add the following lines: ~\\

% code %
\begin{lstlisting}[style=code]
#include <iostream>
using namespace std;
\end{lstlisting}
% code %

                ~\\
               To import libraries of code, we use the \texttt{ \#include } command.

               The C++ Standard Library contains many types of functionality,
               including drawing text to the screen, getting input from the keyboard,
               calculating square roots and trig functions, reading and writing
               text files, and a lot more.
               \textbf{ iostream } allows us to use \texttt{ cin } (console-input)
               and \texttt{ cout } (console-output) in our programs.

               The \texttt{ using namespace std; } line, for the time being,
               we can take for granted. It is basically stating that we're using
               the \textit{ standard } C++ library.

           
            \subsection*{ Add comments }

                In C++, there are two ways to add comments:
                
\begin{lstlisting}[style=code]
// single line
\end{lstlisting}

or

\begin{lstlisting}[style=code]
/*
Multi-
Line
Comment !
*/
\end{lstlisting}

                At the top of your source file, add a comment
                with your name.

            \newpage
            \subsection*{ Program starting point }

                Every program in C++ begins at the \textbf{ main } function.
                For now, we will just memorize how this part of the code looks.
                Later on, we will write our own functions. \\
                
% code %
\begin{lstlisting}[style=code]
int main()
{
    return 0;
}
\end{lstlisting}
% code %

                ~\\
                \textbf{ main } is the name of the function. Every function
                opens and closes with \{ and \}, respectively. Any code
                between the opening and closing curly braces are
                \textit{ inside } the function.

                The \texttt{ return 0; } command is where our program ends -
                the 0 signifies, ``no errors occurred during the program execution".

                While working in main() for now, your program code will go
                below the \{ and above the \texttt{ return 0; }.

            \subsection*{ Displaying output }

                To write text to the screen in C++, we use the \textbf{ cout } command.
                It will look like the following, and go within the \texttt{ main() }
                function, after the opening curly brace \{ and before the
                \texttt{ return 0; }. ~\\

% code %
\begin{lstlisting}[style=code]
int main()
{
    cout << "Hello, world!";
    return 0;
}
\end{lstlisting}
% code %

                ~\\
                In C++, \textbf{ statements } end with a semicolon \texttt{ ; }.
                The semicolon is how C++ knows you're done with a command.

                \newpage
                Commands can span several lines, as long as the semicolon
                is there at the end. In the following example, the \texttt{ cout }
                statement spans five lines, and ends at \texttt{ $<<$ endl; } \\

% code %
\begin{lstlisting}[style=code]
int main()
{
    cout
        << "Hello"
        << endl
        << "World!"
        << endl;
        
    return 0;
}
\end{lstlisting}
% code %

                asdf
            
            
                        % subparagraph
                    % paragraph
                % subsubsection
            % subsection
        % section
    % chapter



\end{document}
