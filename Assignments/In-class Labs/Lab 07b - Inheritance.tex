\documentclass[a4paper,12pt]{book}
\usepackage[utf8]{inputenc}
\author{Rachel Morris}
\date{\today}
\title{ Lab 07b }

\usepackage{rachwidgets}
\usepackage{fancyhdr}
\usepackage{lastpage}
\usepackage{boxedminipage}

\pagestyle{fancy}
\fancyhf{}
\lhead{CS 200}
\chead{Summer 2017}
\rhead{Lab 07b}
\rfoot{\thepage\ of \pageref{LastPage}}
\lfoot{By Rachel Morris \date}

\renewcommand{\headrulewidth}{2pt}
\renewcommand{\footrulewidth}{1pt}

\begin{document}    
    \chapter*{CS 200 Lab 07b: Inheritance} \stepcounter{chapter}
        \section*{Topics}
        
            \begin{itemize}
                \item Classes
                \item Inheritance
            \end{itemize}

        \section*{Starting out}

            Make sure to download the following code files:

            \begin{itemize}
                \item main.cpp
                \item Question.hpp
                \item Question.cpp
                \item Quizzer.hpp
                \item Quizzer.cpp
            \end{itemize}

            Starting off, we won't use all the files at first.

            Create a new project, and only import \textbf{ Question.hpp }
            and \\ \textbf{ Question.cpp }, and create \textbf{ main.cpp } with
            just the standard program starting point.

            Start off with the code on the next page.

            \newpage            
                \subsubsection*{Question.hpp}
\begin{lstlisting}[style=code]
#ifndef _QUESTION_HPP
#define _QUESTION_HPP

#include <string>
#include <iostream>
using namespace std;

class Question
{
};

class TrueFalseQuestion : public Question
{
};

class MultipleChoiceQuestion : public Question
{
};

#endif
\end{lstlisting}

            
                \subsubsection*{Question.cpp}
\begin{lstlisting}[style=code]
#include "Question.hpp"
\end{lstlisting}

            
                \subsubsection*{main.cpp}
\begin{lstlisting}[style=code]
#include <iostream>
#include <string>
using namespace std;

#include "Question.hpp"

int main()
{
    return 0;
}

\end{lstlisting}

            \newpage
            \subsection*{The Question family}

            First, we will build out the \textbf{ Question } base class.

            We aren't going to create any Question objects, but it is the
            starting point for all other questions - this class will contain
            that which is in common to all other Questions.

                    \paragraph{ Question class }
                    
                        \subparagraph{ Member variables } ~\\
                            
                            % table %
                            \begin{tabular}{ l l l l }
                                Accessibility & Data Type & Variable Name & Info \\ \hline{}
                                protected & string & m\_questionText & Used to store the question
                            \end{tabular}
                            % table %
                        
                        \subparagraph{ Member functions } ~\\

                            % table - functions %
                            \begin{tabular}{ l l l l l }
                                Accessibility & Return Type & Function Name & Parameters & Info \\ \hline{}
                                public & void & Display & - & Used to display the question \\ \hline{}
                                public & void & SetQuestionText & string text & Used to set m\_questionText
                            \end{tabular}
                            % table - functions %

                        \subparagraph{ void Display() } ~\\
                            Use a \texttt{ cout } statement to display
                            the value of \texttt{ m\_questionText }.


                        \subparagraph{ void SetQuestionText( string text ) } ~\\

                            This function will be reused by the child classes.
                            Write an assignment statement that will set the
                            member variable, \texttt{ m\_questionText },
                            to the value of the parameter passed in.
                            
                            % framed box %
                            \begin{mdframed}[backgroundcolor=error] 
                                \textbf{ Common mistake }

                                Make sure that you're not \textit{ redeclaring }
                                the variable \texttt{ m\_questionText } -- You shouldn't
                                be using its data type here. \\

                                \texttt{ m\_questionText } is already a \textbf{ member }
                                of the Question class. Just use it by name.
                            \end{mdframed}
                            % framed box %

                            
                            % framed box %
                            \begin{mdframed}[backgroundcolor=hint] 
                            \textbf{ Assignment statement }

                            VARIABLE = VALUE;
                            \end{mdframed}
                            % framed box %

                    \newpage{}
                    \paragraph{ Testing } ~\\
                        Before continuing, test your Question class!

                        Within \texttt{ main() }, declare a Quesiton object,
                        and make sure that both \texttt{ SetQuestion } and
                        \texttt{ Display } function calls work correctly.
                    
                    \paragraph{ TrueFalseQuestion class } ~\\
                        This question type will display the question,
                        then ask the user to enter \textit{ true } or
                        \textit{ false }, and then figure out if they
                        answered correctly.

                        This means that we need functions to set whether
                        \textit{ true } or \textit{ false } is the correct
                        answer, as well as to check the user's response.

                        \subparagraph{ Member variables } ~\\
                            
                            % table %
                            \begin{tabular}{ l l l l }
                                Accessibility & Data Type & Variable Name \\ \hline{}
                                private & string & m\_correctAnswer 
                            \end{tabular}
                            % table %
                        
                        \subparagraph{ Member functions } ~\\

                            % table - functions %
                            \begin{tabular}{ l l l l l }
                                Accessibility & Return Type & Function Name & Parameters  \\ \hline{}
                                public & void & Display & -  \\ \hline{}
                                public & void & SetCorrectAnswer & string correctAnswer \\ \hline{}
                                public & bool & CheckAnswer & string userAnswer
                            \end{tabular}
                            % table - functions %

                        \subparagraph{ void Display() } ~\\

                            This function should call the parents' version of
                            the \texttt{ Display } function first, then add
                            its own unique code in.
                            
% code box %
\begin{lstlisting}[style=code]
void TrueFalseQuestion::Display()
{
    Question::Display();
}
\end{lstlisting}
% code box %
                            Afterward, display another message, asking, \\
                            \textit{ "True or false?" }

                        \newpage
                        \subparagraph{ void SetCorrectAnswer( string correctAnswer ) }  ~\\

                            This function is responsible for assigning the \textit{ value }
                            of the parameter \texttt{ correctAnswer } to the member variable
                            \texttt{ m\_correctAnswer }.


                        \subparagraph{ bool CheckAnswer( string userAnswer ) } ~\\

                            This function receives the answer that the user gave,
                            as the parameter \texttt{ userAnswer }.

                            This function should compare \texttt{ userAnswer } to
                            the member variable \texttt{ m\_correctAnswer } in order
                            to decide if the user was correct or not.

                            \begin{itemize}
                                \item If \texttt{ userAnswer } and \texttt{ m\_correctAnswer } match, then return \texttt{ true }.
                                \item Otherwise, return \texttt{ false }.
                            \end{itemize}
                        
                    \paragraph{ Testing } ~\\
                        Before continuing, test your TrueFalseQuestion class!

                        Within \texttt{ main() }, declare a TrueFalseQuestion object.
                        Use \texttt{ SetQuestionText } to set its question and
                        \texttt{ SetCorrectAnswer } to set the correct answer,
                        then use \texttt{ Display } to view the question text,
                        and \texttt{ CheckAnswer } to see if it correctly detects
                        right and wrong answers. \newpage

% code box %
\begin{lstlisting}[style=code]
int main()
{
    TrueFalseQuestion tfQuestion;

    tfQuestion.SetQuestionText( "Is Kansas a state?" );
    tfQuestion.SetCorrectAnswer( "true" );

    tfQuestion.Display();
    string answer;
    cin >> answer;

    bool result = tfQuestion.CheckAnswer( answer );

    if ( result == true )
        cout << "Correct answer!" << endl;
    else
        cout << "Wrong answer!" << endl;

    return 0;
}
\end{lstlisting}
% code box %

% program output %
\begin{lstlisting}[style=output]
Is Kansas a state?
(true/false): true
Correct answer!
\end{lstlisting}
% program output %

                    \newpage
                    \paragraph{ MultipleChoiceQuestion class } ~\\
                        This type of question will display four options
                        for the user to choose from. The user will select
                        1, 2, 3, or 4, and only one answer will be right.
                        
                        This means that we need functions to set the
                        text for options 1, 2, 3, and 4, as well as
                        store whether option 1, 2, 3, or 4 is the
                        correct answer.

                        \subparagraph{ Member variables } ~\\
                            
                            % table %
                            \begin{tabular}{ l l l l }
                                Accessibility & Data Type & Variable Name \\ \hline{}
                                private & string array, size 4 & m\_answers \\ \hline{}
                                private & int & m\_correctAnswer
                            \end{tabular}
                            % table %
                        
                        \subparagraph{ Member functions } ~\\

                            % table - functions %
                            \begin{tabular}{ l l l l l }
                                Accessibility & Return Type & Function Name & Parameters  \\ \hline{}
                                public & void & Display & -  \\ \hline{}
                                public & void & SetAnswerChoices & string a, string b, \\
                                & & & string c, string d \\ \hline{}
                                public & void & SetCorrectAnswer & int correctAnswer \\ \hline{}
                                public & bool & CheckAnswer & int userAnswer
                            \end{tabular}
                            % table - functions %


                        
                        \subparagraph{ void SetAnswerChoices( string a, string b, string c, string d ) } ~\\
                            Set each of the elements of the array \texttt{ m\_answers }
                            to one of the parameters. \\

                            \begin{tabular}{ c c }
                                Answer \#0 = a &
                                Answer \#1 = b \\
                                Answer \#2 = c &
                                Answer \#3 = d
                            \end{tabular}
                            
                        \subparagraph{ void Display() } ~\\
                            Once again, call the \texttt{ Question } class' version of
                            \texttt{ Display() }, and then use a \textbf{ for loop }
                            to iterate over all 4 options in the member array,
                            \texttt{ m\_answers }, displaying them to the screen.
                        
                        \subparagraph{ void SetCorrectAnswer( int correct ) } ~\\
                            The parameter \texttt{ correct } stores the \textit{ index }
                            of the \texttt{ m\_answers } element that is storing
                            the correct answer.

                            Store this value in the \texttt{ m\_correctAnswer } member variable.

                        \newpage
                        \subparagraph{ bool CheckAnswer( int userAnswer ) } ~\\

                            This function receives the answer that the user gave,
                            as the parameter \texttt{ userAnswer }.

                            This function should compare \texttt{ userAnswer } to
                            the member variable \texttt{ m\_correctAnswer } in order
                            to decide if the user was correct or not.

                            \begin{itemize}
                                \item If \texttt{ userAnswer } and \texttt{ m\_correctAnswer } match, then return \texttt{ true }.
                                \item Otherwise, return \texttt{ false }.
                            \end{itemize}
                        
                    \paragraph{ Testing } ~\\
                        Write a test that creates a MultipleChoiceQuestion,
                        sets the question with \texttt{ SetQuestionText },
                        sets the answer choices with \texttt{ SetAnswerChoices },
                        sets the correct answer with \texttt{ SetCorrectAnswer },
                        displays the question with \texttt{ Display },
                        and checks if the user's answer was right with \texttt{ CheckAnswer }. \\

                    \newpage
% code box %
\begin{lstlisting}[style=code]
int main()
{
    MultipleChoiceQuestion mcQuestion;

    mcQuestion.SetQuestionText( "What is the capital of Kansas?" );
    mcQuestion.SetAnswerChoices( "Topeka", "Wichita",
                                "Kansas City", "Boise" );
    mcQuestion.SetCorrectAnswer( 0 );

    mcQuestion.Display();
    int answer;
    cout << ">> ";
    cin >> answer;

    bool result = mcQuestion.CheckAnswer( answer );

    if ( result == true )
        cout << "Correct answer!" << endl;
    else
        cout << "Wrong answer!" << endl;

    return 0;
}
\end{lstlisting}
% code box %

% program output %
\begin{lstlisting}[style=output]
What is the capital of Kansas?

OPTIONS:
0. Topeka
1. Wichita
2. Kansas City
3. Boise
>> 2
Wrong answer!

\end{lstlisting}
% program output %
                    
            \newpage{}
            \subsection*{The full program}

                Now that your Question classes are working,
                import in the \textbf{ Quizzer.hpp } and \textbf{ Quizzer.cpp }
                files to your project, and overwrite \textbf{ main.cpp }
                with the file provided.


            \subsection*{Quizzer.hpp}
\begin{lstlisting}[style=code]
#ifndef _QUIZZER_HPP
#define _QUIZZER_HPP

#include "Question.hpp"

class Quizzer
{
public:
    Quizzer();

    void AddTrueFalseQuestion( TrueFalseQuestion* q );
    void AddMultipleChoiceQuestion( MultipleChoiceQuestion* q );

    void Run();

private:
    TrueFalseQuestion* m_tfQuestions[3];
    MultipleChoiceQuestion* m_mcQuestions[3];

    int m_count_tfQuestions;
    int m_count_mcQuestions;
};
\end{lstlisting}

            
            \subsection*{Quizzer.cpp}
\begin{lstlisting}[style=code]
#include "Quizzer.hpp"

#include <iostream>
#include <string>
using namespace std;

Quizzer::Quizzer()
{
    m_count_mcQuestions = 0;
    m_count_tfQuestions = 0;
}

void Quizzer::AddTrueFalseQuestion( TrueFalseQuestion* q )
{
    if ( m_count_tfQuestions >= 3 ) { return; }
    m_tfQuestions[ m_count_tfQuestions++ ] = q;
}

void Quizzer::AddMultipleChoiceQuestion( MultipleChoiceQuestion* q )
{
    if ( m_count_mcQuestions >= 3 ) { return; }
    m_mcQuestions[ m_count_mcQuestions++ ] = q;
}

void Quizzer::Run()
{
    int totalQuestions = m_count_tfQuestions + m_count_mcQuestions;
    int totalRight = 0;

    for ( int i = 0; i < m_count_tfQuestions; i++ )
    {
        m_tfQuestions[ i ]->Display();

        string answer;
        cin >> answer;

        bool correct = m_tfQuestions[i]->CheckAnswer( answer );

        if ( correct )
        {
            cout << "CORRECT!" << endl;
            totalRight++;
        }
        else
        {
            cout << "INCORRECT!" << endl;
        }
    }

    for ( int i = 0; i < m_count_mcQuestions; i++ )
    {
        m_mcQuestions[ i ]->Display();

        int answer;
        cin >> answer;

        bool correct = m_mcQuestions[i]->CheckAnswer( answer );

        if ( correct )
        {
            cout << "CORRECT!" << endl;
            totalRight++;
        }
        else
        {
            cout << "INCORRECT!" << endl;
        }
    }

    cout << endl << endl;
    cout << "Final Score: " << totalRight << " out of " << totalQuestions << endl;
}
\end{lstlisting}

            \newpage                    
            \subsection*{main.cpp}
\begin{lstlisting}[style=code]
#include <iostream>
using namespace std;

#include "Quizzer.hpp"

int main()
{
    Quizzer quizzer;

    TrueFalseQuestion tf1, tf2, tf3;
    tf1.SetQuestionText( "Static arrays can be resized at run-time." );
    tf1.SetCorrectAnswer( "false" );
    quizzer.AddTrueFalseQuestion( &tf1 );

    tf2.SetQuestionText( "Classes can contain member variables and functions." );
    tf2.SetCorrectAnswer( "true" );
    quizzer.AddTrueFalseQuestion( &tf2 );

    tf3.SetQuestionText( "It is good practice to set a pointer to nullptr when not in use." );
    tf3.SetCorrectAnswer( "true" );
    quizzer.AddTrueFalseQuestion( &tf3 );

    MultipleChoiceQuestion mc1, mc2, mc3;
    mc1.SetQuestionText( "Which of the following is the address-of operator?" );
    mc1.SetAnswerChoices( "&", "*", "->", "::" );
    mc1.SetCorrectAnswer( 0 );
    quizzer.AddMultipleChoiceQuestion( &mc1 );

    mc2.SetQuestionText( "Dynamic variables are allocated on the..." );
    mc2.SetAnswerChoices( "stack", "heap", "queue", "array" );
    mc2.SetCorrectAnswer( 1 );
    quizzer.AddMultipleChoiceQuestion( &mc2 );

    mc3.SetQuestionText( "When a value is being passed into a function call, it is known as a..." );
    mc3.SetAnswerChoices( "parameter", "structure", "reference", "argument" );
    mc3.SetCorrectAnswer( 3 );
    quizzer.AddMultipleChoiceQuestion( &mc3 );

    quizzer.Run();

    return 0;
}
\end{lstlisting}

            \newpage{}
            \subsection*{ Run and test }

                Run the program and make sure it works with your code

% program output %
\begin{lstlisting}[style=output]
Static arrays can be resized at run-time.
(true/false): false
CORRECT!

Classes can contain member variables and functions.
(true/false): true
CORRECT!

It is good practice to set a pointer to nullptr when not in use.
(true/false): true
CORRECT!

Which of the following is the address-of operator?

OPTIONS:
0. &
1. *
2. ->
3. ::
0
CORRECT!

Dynamic variables are allocated on the...

OPTIONS:
0. stack
1. heap
2. queue
3. array
1
CORRECT!

When a value is being passed into a function call, it is known as a...

OPTIONS:
0. parameter
1. structure
2. reference
3. argument
3
CORRECT!


Final Score: 6 out of 6
\end{lstlisting}
% program output %

                        % subparagraph
                    % paragraph
                % subsubsection
            % subsection
        % section
    % chapter




\end{document}
